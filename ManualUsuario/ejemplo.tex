%--------Se especifica el tipo de documento
\documentclass[oneside,10pt]{book}
%--------------Hojas de Estilos
\usepackage{DEBook}
\usepackage{userManual}
%-------- Datos del documento
\title{Sistema De BiBlioteca}
\subtitle{Biblioteca Central}
\author{3CM2}
\organization{Escuela Superior de Cómputo, IPN}
%%%%%%%%%%%%%%%%%%%%%%%%%%%%%%%%%%%%%%%%%%%%%%%%%%%%%%%%%%%%%%%%
\begin{document}
\maketitle
\frontmatter
\tableofcontents
\mainmatter
%%=========================================================
\chapter{Uso General Del Sistema}
\section{Entorno De Trabajo}
En esta sección se describe el entorno de trabajo del sistema, se explica la disposición
de los elementos principales y comunes de las pantallas, los colores, la iconografía, componentes, etc.
utilizados dentro del sistema y el uso general del sistema.
\subsection{Diseño}
\UMDesignFigure{designexample}{Ejemplo de figura}
\subsection{Iconografía}
En el sistema se utilizan iconos para denotar las diferentes operaciones que el usuario puede realizar
en el sistema. Los iconos utilizados se describen a continuación:
Nota: Desarrollo tiene que tener sus íconos para que el equipo de plantillas pueda generar el comando específico
\begin{Iconography}
	\item \ICexample Icono de Ejemplo
\end{Iconography}
\subsection{Organización}
Las funcionalidades del sistema se encuentran organizadas por menús. Cada perfil de usuario accede
a un menú diferente ya que este le describe su ciclo de trabajo y las funciones que puede realizar.
\OMenu{Menú de Ejemplo}{example}{Menú de ejemplo}
Se describe el menú, de manera que indique en qué, para qué y porqué se usa.
\subsection{Componentes Utilizados}
En esta sección se explican los diferentes componentes que se utilizan para capturar o mostrar
información en el sistema.
\OComponent{Componente de Ejemplo}{example}{Componente de ejemplo}
Se describe el componente.
\chapter{Información General Del Proceso}
Este capítulo explica el proceso que se lleva a cabo.
\subsection{Información 1}
Aquí se describe la información relacionada a un actor o a los documentos/entradas que este necesite para el proceso.
\chapter{Sub Paso 1 de Proceso X}
\begin{UMStep}{Paso de Ejemplo}
	\item Elemento 1
	\item Elemento 2
\end{UMStep}
\end{document}