En el sistema se utilizan iconos para denotar las diferentes operaciones que el usuario puede realizar
en el sistema. Los iconos utilizados se describen a continuación:
Nota: Desarrollo tiene que tener sus íconos para que el equipo de plantillas pueda generar el comando específico
\begin{Iconography}
	\item \ICexample Icono de Ejemplo
	\item \ICCancel  Permite cancelar un proceso en el momento que el usuario lo desee.
	\item \ICConfirm Permite confirmar los datos ingresados por el usuario.
	\item \ICQuery	 Permite generar y consultar estadísticas con base en los préstamos efectuados.
	\item \ICUnlockA Icono principal de inicio de sesión
	\item \ICEmail	Hace referencia al campo de correo electrónico
	\item \ICManageLoan  Accede a la gestión de préstamos
	\item \ICManageResource Accede a la gestión de recursos
	\item \ICTime	Hace referencia a los horarios disponibles de reservación
	\item \ICHome	Realiza una referencia a la pagina de inicio o home
    \item \ICName	Denota el campo o apartado de texto
	\item \ICPassword	Denota el campo o apartado de password
	\item \ICBooking Hace referencia a un calendario, donde se podrá seleccionar una fecha
	\item \ICStudyRoom  Hace referencia a una sala de estudio.
	\item \ICExpRoom  Hace referencia a una sala de exposiciones.
	\item \ICWorkRoom Hace referencia a una sala de trabajo
	\item \ICLogOut	Permite cerrar la sesión del usuario en cuestión.
	\item \ICUser   Hace referencia a la cuenta del usuario.
\end{Iconography}