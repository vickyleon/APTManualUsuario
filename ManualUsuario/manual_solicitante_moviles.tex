\documentclass[oneside,10pt,letterpaper]{book}

\usepackage{formato_cdt}          % El rollo de la CDT
\newcommand{\Titulo}{Manual de usuario: solicitante}
\usepackage{formato_propio}       % Nuestro rollo
\usepackage{manual_de_usuario}

\begin{document}
\maketitle
\frontmatter
\tableofcontents
\mainmatter
%%=========================================================
\chapter{Uso General Del Sistema}
\section{Entorno De Trabajo}
En esta sección se describe el entorno de trabajo del sistema, se explica la disposición
de los elementos principales y comunes de las pantallas, los colores, la iconografía, componentes, etc.
utilizados dentro del sistema y el uso general del sistema.

\subsection{Diseño}
	\UMDesignFigure{inicio_Solicitante}{Inicio de Solicitante}
Se puede observar en la parte superior, en el encabezado de la página,
el menú del solicitante donde éste puede regresar a la pantalla de inicio,
salir o configurar su cuenta. En esta opción se puede cambiar la contraseña
y el correo si se desea.

Más abajo, encontramos el perfil del usuario, donde se mostrará el nombre
del mismo. 

Para hacer las reservaciones y otras funciones, todo se mostrará en la zona
de trabajo.


\subsection{Iconografía}
	En el sistema se utilizan iconos para denotar las diferentes operaciones que el usuario puede realizar
en el sistema. Los iconos utilizados se describen a continuación:
Nota: Desarrollo tiene que tener sus íconos para que el equipo de plantillas pueda generar el comando específico
\begin{Iconography}
	\item \ICexample Icono de Ejemplo
	\item \ICCancel  Permite cancelar un proceso en el momento que el usuario lo desee.
	\item \ICConfirm Permite confirmar los datos ingresados por el usuario.
	\item \ICQuery	 Permite generar y consultar estadísticas con base en los préstamos efectuados.
	\item \ICUnlockA Icono principal de inicio de sesión
	\item \ICEmail	Hace referencia al campo de correo electrónico
	\item \ICManageLoan  Accede a la gestión de préstamos
	\item \ICManageResource Accede a la gestión de recursos
	\item \ICTime	Hace referencia a los horarios disponibles de reservación
	\item \ICHome	Realiza una referencia a la pagina de inicio o home
    \item \ICName	Denota el campo o apartado de texto
	\item \ICPassword	Denota el campo o apartado de password
	\item \ICBooking Hace referencia a un calendario, donde se podrá seleccionar una fecha
	\item \ICStudyRoom  Hace referencia a una sala de estudio.
	\item \ICExpRoom  Hace referencia a una sala de exposiciones.
	\item \ICWorkRoom Hace referencia a una sala de trabajo
	\item \ICLogOut	Permite cerrar la sesión del usuario en cuestión.
	\item \ICUser   Hace referencia a la cuenta del usuario.
\end{Iconography}
\subsection{Organización}
	Las funcionalidades del sistema se encuentran organizadas 
en su mayoría en la parte superior derecha de la siguiente manera:

\begin{enumerate}
	\item Inicio
	\item Config
	\item Salir
\end{enumerate}

Exceptuando la función para reservación, localizada en el centro
con el botón \textbf{Reservar}.


\subsection{Componentes Utilizados}
	En esta sección se explican los diferentes componentes que se utilizan para capturar o mostrar
información en el sistema.
\OComponent{Componente de Ejemplo}{example}{Componente de ejemplo}
Se describe el componente.

\chapter{Proceso Iniciar Sesión}
	Este capítulo explica el proceso que se lleva a cabo para ingresar al 
	sistema con un identificador y una contraseña. 
	Se tiene la siguiente interfaz:
	
	\begin{figure}[hbtp]
		
		\includegraphics[scale=0.5]{images/InterfazMovil/IUGS00_login.png}
		\caption{Iniciar sesión}
	\end{figure}

%%\section{Información general de Iniciar sesión}
%%	\subsection{Bienvenida}

Se tiene la interfaz de bienvenida donde se tiene que presionar el botón  \textbf{Gestionar usuarios} ubicado en el menú del lado izquierdo:
	\begin{figure}[hbtp]
		\centering
		\includegraphics[scale=0.3]{images/Interfaz/IUGS22_binevenida.png}
		\caption{Bienvenida Administrador}
	\end{figure}
\subsection{Pantalla Gestionar usuario}
Se muestra la siguiente interfaz donde se gestionan los usuarios

	\begin{figure}[hbtp]
		\centering
		\includegraphics[scale=0.3]{images/Interfaz/IUGS22_gestionarUsuario.png}
		\caption{Gestionar Usuarios}
	\end{figure}

\section{Paso 1. Iniciar sesión}
	Se ingresan los datos de identificación para iniciar una sesión.
\subsection{Subpaso 1-A: Ingresar credenciales}
\begin{itemize}
	\item Ingrese identificador.
	\item Ingrese contraseña.
	\item Presione el botón \textbf{Entrar}. Este paso puede derivar
		en los errores \textbf{Error E1-A} y \textbf{Error E1-B}.
\end{itemize}
Observación 1: un identificador está compuesto por seis letras seguidas
	de cuatro dígitos.
	
	\begin{figure}[hbtp]
		\centering
		\includegraphics[scale=0.3]{images/InterfazMovil/IUGS00_login.png}
		\caption{Iniciar sesión}
	\end{figure}
	

\subsection{Error E1-A: usuario no registrado}
El identificador que se ingresó no se encuentra registrado en el sistema.
\begin{itemize}
	\item Presionar \textbf{Aceptar} en la ventana emergente 
		\textbf{IUGS-30: usuario no registrado}
\end{itemize}

\subsection{Error E1-B: contraseña equivocada}
La contraseña que ingresó no concuerda con el valor que se tiene almacenado.
\begin{itemize}
	\item Presionar \textbf{Aceptar} en la ventana emergente 
		\textbf{IUGS-31: contraseña equivocada}
\end{itemize}


\section{Paso 2. Bienvenida}
	\subsection{Información del solicitante}
\begin{figure}[hbtp]
		\centering
		\includegraphics[scale=0.3]{images/Interfaz/IUGS00_binevenida.png}
		\caption{Bienvenida para Administrador}
	\end{figure}



%%\input{images/interfaces/ApoyoTecnico/pantallas}
\end{document}
